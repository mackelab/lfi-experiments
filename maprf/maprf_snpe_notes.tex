\documentclass[10pt,english]{article}
%\special{papersize=210mm,297mm}
%\usepackage[a4paper,left=20mm,right=20mm, top=2.5cm, bottom=2.5cm]{geometry}
\usepackage{epsfig}
\usepackage{babel}
\usepackage{amssymb}
\usepackage{amsmath}
\usepackage{units}
\usepackage{pstricks}
\usepackage{color}
\usepackage{bm}
\bibliographystyle{plain}

\usepackage{framed}

% Super-handy maths abbreviations
\newcommand{\yb}{\mathbf{y}}
\newcommand{\xb}{\mathbf{x}}
\newcommand{\Xb}{\mathbf{X}}
\newcommand{\pstrxobs}{q_\phi(\theta|x_0)}
\newcommand{\pstrx}{q_\phi(\theta|x)}


\title{mapRF \& SNPE}
\author{}
\date{}

\begin{document}


\maketitle

\section{Sufficient statistics for nonlinearly parametrized filters in GLMs}
\label{seq:sufficiency_STA}

We consider the generalized linear model (GLMs) with Poisson noise distribution $\mathcal{P}(\bullet|\lambda)$, canonical link function $\eta= \log(\lambda)$ and linear predictor $\eta = \beta^\top \xb$. 
Additionally, we assume the linear filter $\beta(\theta)$ itself to be a fixed (nonlinear) function of (lower-dimensional) parameter $\theta$. 
\begin{align}
p(\yb | \Xb, \theta) &= \prod_i p(y_i | \xb_i, \theta) \nonumber \\
&= \prod_i \mathcal{P}\left(y_i | \lambda_i = \exp(\beta(\theta)^\top \xb_i) \right) \nonumber \\
&= \prod_i  \frac{\lambda_i^{y_i}}{y_i!} \exp(-\lambda_i) \nonumber \\
&= \prod_i \frac{1}{y_i!} \exp\left( y_i \beta(\theta)^\top \xb_i - \exp(\beta(\theta)^\top \xb_i) \right) \nonumber \\
&= \frac{1}{\prod_i y_i!} \exp\left( y_i \sum_i \beta(\theta)^\top \xb_i - \exp(\beta(\theta)^\top \xb_i) \right) \nonumber \\
&= \frac{1}{\prod_i y_i!} \exp\left( \beta(\theta)^\top \left(\Xb^\top \yb \right) - \sum_i  \exp(\beta(\theta)^\top \xb_i) \right) \nonumber \\
&=: h(\yb) g_{(\theta, \Xb)}(T(\yb)) \nonumber
\end{align}
with 
\begin{align}
h(\yb) &:= \frac{1}{\prod_i y_i!} \nonumber \\
g_{(\theta, \Xb)}(T(\yb)) &= \frac{1}{Z} \exp\left( \beta(\theta)^\top \left(\Xb^\top \yb \right) \right) \nonumber \\
T(\yb) &= \Xb^\top \yb \nonumber 
\end{align}
Note that this derivation is analoguous to derivations of sufficient statistics for 'standard' GLMs that treat $\beta$ as an independent variable rather than parametrized $\beta(\theta)$.
\begin{align}
p(\theta | \yb, \Xb) = p(\theta | \Xb^\top\yb, \Xb)
\end{align}

\section{SNPE SVI}
\label{seq:SVI_SNPV_vs_CDELFI}

\subsection{Derivation of non-SVI losses for CDELFI and SNPE}
Papamakarios \& Murray start out the derivation of their non-SVI algorithm from a variational approximation to a joint probability over $(\bf{x}, \theta)$ with convencience ('proposal
) prior $\tilde{p}(\theta)$ :
\begin{align}
D_{KL}\left( \ p(\bf{x}|\theta) \ \tilde{p}(\theta) \ || \ \tilde{p}(\bf{x}) \ q_\phi(\theta|\bf{x}) \ \right) &= - E_{p(\bf{x}|\theta)\tilde{p}(\theta)}\left[ \log q_\phi(\theta | \bf{x}) \right] + const. \nonumber \\
&\approx - \sum_n \log q_\phi(\theta_n | \bf{x}_n) \nonumber + const.
\end{align}
for $(\theta_n, \bf{x}_n) \sim \tilde{p}(\theta) p(\bf{x}|\theta)$.
The righthand-side is a variational lower bound that is optimized using stochastic gradient descent. 
Papamakarios \& Murray subsequently analytically correct for having used the wrong prior $\tilde{p}(\theta)$ in the loss, which however only works for certain functional forms for $p, \tilde{p}, q_\phi$.

\paragraph{}\noindent{}SNPE directly optimises the joint probability for the true prior:
\begin{align}
D_{KL}\left( \ p(\bf{x}|\theta) \ p(\theta) \ || \ p(\bf{x}) \ q_\phi(\theta|\bf{x}) \ \right) &= - E_{p(\bf{x}|\theta)p(\theta)}\left[ \log q_\phi(\theta | \bf{x}) \right] + const. \nonumber \\
&= - E_{p(\bf{x}|\theta)\tilde{p}(\theta)}\left[ \frac{p(\theta)}{\tilde{p}(\theta)} \log q_\phi(\theta | \bf{x}) \right] + const. \nonumber \\
&\approx - \sum_n \frac{p(\theta_n)}{\tilde{p}(\theta_n)} \log q_\phi(\theta_n | \bf{x}_n) + const. \nonumber
\end{align}

\subsection{Derivation of SVI losses for CDELFI and SNPE}

The derviation of the CDELFI SVI loss in Papamakarios \& Murray is only sketched (appendix D). 
Filling in between the verbal description of their approach, it seems to rest on 
\begin{align}
D_{KL}( \ q(\phi) \ || \ p(\phi \ | \ \{\bf{x}_n, \theta_n\}) \ ) &= - E_{q(\phi)}\left[ \log p(\phi \ | \ \{\bf{x}_n, \theta_n\}) \right]  + E_{q(\phi)}\left[ \log q(\phi) \right] \nonumber \\
&= - E_{q(\phi)}\left[ \log p( \{\bf{x}_n, \theta_n\} \ | \ \phi) \right]  + D_{KL}( \ q(\phi) \ || \ p(\phi) \ )  + const. \nonumber \\
&= - E_{q(\phi)}\left[ \sum_n \log p( \bf{x}_n, \theta_n \ | \ \phi) \right]  + D_{KL}( \ q(\phi) \ || \ p(\phi) \ )  + const. \nonumber \\
&= - E_{q(\phi)}\left[ \sum_n \log q_\phi( \theta_n | \bf{x}_n) \right]  + D_{KL}( \ q(\phi) \ || \ p(\phi) \ )  + const. \nonumber
\end{align}
where we plugged in the \textbf{joint-data likelihood} $p( \bf{x}_n, \theta_n \ | \ \phi) \propto q_\phi(\theta_n \ | \ \bf{x}_n) \tilde{p}(\bf{x}_n)$. 
This choice of likelihood $p( \bf{x}_n, \theta_n \ | \ \phi)$ is sensible given that we desired
\begin{align}
q_\phi(\theta | \bf{x}) \ \tilde{p}(\bf{x}) \approx p(x |\theta) \ \tilde{p}(\theta) \nonumber
\end{align}
already in the non-SVI case, and that in turn $(\bf{x}_n, \theta_n) \sim p(x|\theta)\tilde{p}(\theta)$ by construction of the artificial dataset.



\noindent{}What for SNPE? The variational bound here (for the first round, and without calibration kernel) reads 
\begin{align}
D_{KL}( \ q(\phi) \ || \ p(\phi \ | \ \{\bf{x}_n, \theta_n\}) \ ) &= - E_{q(\phi)}\left[ \log p( \{ \bf{x}_n, \theta_n \ \} \ | \ \phi) \right]  + D_{KL}( \ q(\phi) \ || \ p(\phi) \ )  + const. \nonumber \\
&= - E_{q(\phi)}\left[ \sum_n \frac{p(\theta)}{\tilde{p}(\theta)}\log q_\phi( \theta_n | \bf{x}_n) \right]  + D_{KL}( \ q(\phi) \ || \ p(\phi) \ )  + const. \nonumber
\end{align}
for the functional forms of $p(\phi)$ and $q(\phi)$ as above.
Note that we still draw data from the proposal prior, i.e.
\begin{align}
(\bf{x}_n, \theta_n) \sim p(x|\theta)\tilde{p}(\theta),\nonumber
\end{align} 
but for SNPE we actually want to directly approximate the real posterior,
\begin{align}
q_\phi(\theta \ | \ \bf{x}) p(\bf{x}) \approx p(x |\theta) p(\theta). \nonumber
\end{align}
Thus the task here is to find a likelihood $p( \bf{x}_n, \theta_n \ | \ \phi)$ that bridges this gap: 
on one hand being a good model for the data $(\bf{x}_n, \theta_n) \sim p(\bf{x}|\theta)\tilde{p}(\theta)$, but on the other hand being useful/adjustable to model $p(\bf{x} |\theta) p(\theta)$.

Reading the likelihood off the variational bound however, we find
\begin{align}
p( \bf{x}_n, \theta_n \ | \ \phi) \propto q_\phi(\theta_n \ | \ \bf{x}_n)^\frac{p(\theta_n)}{\tilde{p}(\theta_n)}
\label{eq:SNPE_likelihood}
\end{align}

\begin{framed}
\noindent{}\textbf{Remark:} What exactly does such a likelihood correspond to? 

Generally, is it even a proper likelihood? 
Take for instance a uniform prior on scalar $\theta$ with $p(\theta) = \frac{1}{b-a}$ for $\theta \in [a,b]$ and finite $a,b\in\mathbb{R}, b>a$, and $p(\theta) = 0$ otherwise. 
If the proposal prior has infinite support (as is the case if it is Gaussian), then the likelihood for fixed $\phi$ has density$p(\bf{x}, \theta \ | \ \phi) = 1$
for $(\bf{x}, \theta)$ with $\theta \notin [a,b]$.
Since $a,b$ are finite, $p(\bf{x}, \theta \ | \ \phi)$ in this case cannot be normalized over the full support $(\bf{x}, \theta) \in \mathbb{R}^{dim(\bf{x})} \times \mathbb{R}$. 

Similarly, if $p(\theta), \tilde{p}(\theta)$ and $q_\phi(\theta|\bf{x})$ all are univariate Gaussians, one can show that $p(\bf{x},\theta | \phi)$ becomes unnormalizable (for any fixed $\bf{x}$) if the proposal $\tilde{p}$ is broader than the prior $p$, i.e. $\tilde{\sigma}^2 > \sigma^2$:
\begin{align}
q_\phi(\theta | \bf{x})^\frac{p(\theta)}{\tilde{p}(\theta)} &\propto \exp c \exp \left( 2 \log \frac{|\theta - \mu_{\phi}|}{\sigma_{\phi}} + \left(\frac{\mu}{\sigma} - \frac{\tilde{\mu}}{\tilde{\sigma}}\right) \theta - \left(\sigma^{-2} - \tilde{\sigma}^{-2} \right) \theta^2 \right) \nonumber, \\
q_\phi(\theta | \bf{x})^\frac{p(\theta)}{\tilde{p}(\theta)} &\rightarrow 1 \mbox{ for } \theta \rightarrow \pm \infty \nonumber \mbox{ and } \tilde{\sigma}^{-2} < \sigma^{-2}
\end{align}
where $c \leq 0$ is a constant, and parameters $\mu, \tilde{\mu}, \mu_\phi$ correspond to prior $p$, proposal prior $\tilde{p}$ and posterior approximation $q_\phi$, respectively.
While this might not be of high practical relevance in most cases (we usually want to use proposals that are narrower than the prior), this adds an another constraint that was not there before. 
When the observed data does not constrain a certain parameter (i.e. after converence we should have $\tilde{\sigma} = \sigma$, we might indeed get $\tilde{\sigma} > \sigma$ due to random fluctuations in the artificial data set.

An immediate next question is whether we need any (normalization) constraints on $p(\theta)/\tilde{p}(\theta)$ for eq. \ref{eq:SNPE_likelihood} to be a valid likelihood - if such constraints exist, they could lead to sensible normalization factors for the importance weights. 
Note that any such normalization factor $Z$ would show up as a sort of global temperature, as in $p( \bf{x}_n, \theta_n \ | \ \phi)^{\frac{1}{Z}}$
\end{framed}

\begin{framed}
\noindent{}\textbf{Remark:} An alternative likelihood to eq. \ref{eq:SNPE_likelihood} one might consider is 
\begin{align}
p( \bf{x}_n, \theta_n \ | \ \phi) \propto q_\phi(\theta_n \ | \ \bf{x}_n) \frac{p(\theta_n) \ \tilde{p}(\bf{x}_n)}{\tilde{p}(\theta_n) \ p(\bf{x}_n)},
\end{align}
since it is the density that translates between the known real and proposal priors. 
The multiplicative factors $\frac{p(\theta_n) \ \tilde{p}(\bf{x}_n)}{\tilde{p}(\theta_n) \ p(\bf{x}_n)}$ however are just constant factors in the loss that have no influence on the optimization results - we would recover the exact same solution as CDELFI and still need to explicitly correct for the proposal prior post-learning. 
\end{framed}
\end{document}
